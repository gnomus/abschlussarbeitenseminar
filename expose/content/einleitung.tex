
\section{Einleitung}





In den letzten Jahren ist die Anzahl der dynamischen Webapplikationen mit einem hohen Anteil an Javascript stetig gewachsen \quelle.
Allein die wohl bekannteste Javascript Biblothek jQuery \urlref wird heute auf über XX Prozent der Webseiten weltweit eingesetzt. \quelle

Gerade hochdynamische Webanwendungen die mit Bibliotheken wie AngularJS \urlref entwickelt, werden haben ein hohes Bedürfnis an gesicherter Kommunikation.
Während die Web-Technologie sich mit großen Schritten weiterentwickelt ist die der gesicherten Kommunikation umso weniger vorangeschritten.
Seit Jahren besteht die einzige Möglichkeit die Datenintegrität von Webseiten und Javascript zu gewährleisten in SSL Zertifikaten \isref{subsec:ssl}.
Dieser Monopolstellung sind sich die Zertifikatsanbieter natürlich bewusst und dementsprechend ist auch das Angebot.
SSL-Zertifikate sind so teuer, dass es für vereinzelte Personen oder Studenten nahezu unmöglich ist sich die Sicherheit zu leisten.

\todo[inline]{- Aktuelle Webapps immer javascript lastiger\\
- Sicherheit gegen Mitm Attacken nur durch ssl\\
- ssl ist teuer \\
- der einfach mann kann sich kein teures ssl zertifikat leisten \\
- entwickeln einer lösung die ohne zentrale signing authority auskommt \\
}

\subsection{Motivation}
Zum Verfassungszeitpunkt dieser Arbeit richtete sich mein aktueller Fokus in der IT-Welt auf die Themen Security und Web 2.0, daher lag es nahe entsprechend Ausgerichtetes Thema zu wählen.

Weiterhin habe ich mich schon oft über die herrschenden Misstände im Bereich SSL-Zertifikate geärgert und bin schon lange der Meinung, dass eine kostengünstige Alternative zu überteuerten SSL-Zertifikaten geschafft werden muss.

\todo[inline]{- aktueller fokus auf webentwicklung\\
- starkes interesse an security\\
- problem mit zu teuren zertifikaten\\
}

\subsection{Zielsetzung}

In dieser Arbeit untersuche ich, welche Möglichkeiten existieren um die aktuellen Lage zu verbessern. Es soll unter der Annahme verschiedener Rahmenbedingungen schrittweise ein sicheres und unabhängiges Konzept zur Gewährleistung von Datenintigrität in Webapplikationen entwickelt werden.

Abschließend wird das Entwickelte Konzept auf seine Tauglichkeit im realen Anwendungskontext evaluiert und ein Ausblick in die Zukunft der Browserkryptographie gegeben.


\todo[inline]{- betrachten der schwachstellen im aktuellen system\\
- beurteilung des aktuellen standes von browser crypto \\
- theoretisches entwickeln eines konzeptes für das kryptographische signieren von javascript code
}

\subsection{Vorgehensweise und Aufbau der Arbeit}
\tbd

\todo[inline]{- betrachten verschiedener szenarien \\
- von einfach zu schwer \\
- zuerst mit zentraler signing db \\
- dann mit dezentraler signing db \\
- danach autark ohne signing db
}
