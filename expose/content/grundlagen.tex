\section{Grundlagen}

Dieses Kapitel befasst sich mit der Erläuterung der dieser Arbeiten zugrunde liegenen Technologien sowie ihrem aktuellen Stand in der heutigen Software-Welt.
Dem Leser soll genug Handwerkszeug mit auf den Weg gegeben werden um die folgenden Kapitel auch ohne tiefgreifende IT-Kenntnisse zu erfassen und den Gedankengängen des Authoren ohne große Mühe zu folgen.

\subsection{SSL/TLS}
\label{subsec:ssl}
\todo[inline]{- grafik\\
- historie, verbreitung\\
- schwachstellen\\
- CAs\\
- lets encrypt
}


\subsection{Man-in-the-Middle Angriff}

Der Man-in-the-Middle Angriff ist eine Attacke auf die Kommunikation zweier oder mehrerer Teilnehmer eines Computernetzwerkes. Der Angriff basiert auf der Kompromittierung der Netzwerkinfrastruktur zwischen den Kommunikationspartnern.\ifref{fig:mitm-safe}

Hierbei reicht es bereits für den Angreifer Kontrolle über einen einzelnen Netzwerkknoten zu erlangen über den die Kommunikation läuft.
Hat der Angreifer einen solchen Knoten übernommen ist er in der Lage jede eingehende Nachricht mitzulesen (passiver MitM) und ggf. zu verändern (aktiver MitM).

Dies kann dazu führen, dass geheime Daten (z.B. Anmelde-Informationen) mitgeschnitten werden oder dem Opfer kompromittierte  Software untergeschoben wird.

Als Absicherung gegen passive MitM Angriffe wird überlicherweise Verschlüsselung eingesetzt. Zwar ist Verschlüsselung auch ein Mittel gegen aktive MitM Angriffe, besser jedoch ist hier die Signierung von Nachrichten.


\begin{figure}[ht]
  \centering
  \begin{tikzpicture}

    \node[cloud, cloud puffs=12.7, cloud ignores aspect, minimum width=6cm, minimum height=4cm, align=center, draw] (cloud) at (5, 0) {Netzwerk};

    \node[draw] (alice) at (0,0) {Alice};
    \node[draw] (bob) at (10,0) {Bob};
    \node[draw, fill=red!50] (eve) at (5,3) {Eve};

    \draw[thick, ->] (alice) to [bend left=20] node [midway, fill=green!10, draw, thin, rounded corners=1mm] {Hallo Bob! } (bob) ;
    \draw[thick, ->] (bob) to [bend left=20] node [midway, fill=green!10, draw, thin, rounded corners=1mm] {Hallo Alice! } (alice);

  \end{tikzpicture}
  \caption{Kommunikation ohne Man-in-the-Middle}
  \label{fig:mitm-safe}
\end{figure}

\begin{figure}[ht]
  \centering
  \begin{tikzpicture}

    \node[cloud, cloud puffs=12.7, cloud ignores aspect, minimum width=6cm, minimum height=4cm, align=center, draw, fill=red!25] (cloud) at (5, 0) {};

    \node[draw] (alice) at (0,0) {Alice};
    \node[draw] (bob) at (10,0) {Bob};
    \node[draw, fill=red!50] (eve) at (5,0) {Eve};

    \draw[thick, ->] (alice) to [bend left=30] node [midway, fill=green!10, draw, thin, rounded corners=1mm] {Hallo Bob! } (eve) ;
    \draw[thick, ->] (eve) to [bend left=30] node [midway, fill=red!40, draw, thin, rounded corners=1mm] {Bob ist doof! } (bob) ;

    \draw[thick, ->] (bob) to [bend left=40] node [midway, fill=green!10, draw, thin, rounded corners=1mm] {Hey, das ist gemein! } (eve);
    \draw[thick, ->] (eve) to [bend left=40] node [midway, fill=red!40, draw, thin, rounded corners=1mm] {Hallo Alice! } (alice);

  \end{tikzpicture}
  \caption{Man in the Middle Attack}
  \label{fig:mitm-danger}
\end{figure}


\todo[inline]{- grafik\\
- erläuterung anhand von beispielen
}

\subsection{Unterschied Verifikation und Validierung}
\tbd

\subsection{Proof-Carrying Code}
\todo[inline]{- was ist pcc\\
- wo wird er eingesetzt
- vorteile/nachteile
}


\subsection{Kryptographische Signaturmethoden}
\todo[inline]{- pgp\\
- ???
}

\subsection{Kryptographische APIs in aktuellen Webbrowsern}
\todo[inline]{- chrome, firefox, safari, internet explorer\\
- aktueller stand der apis
- was fehlt?
}
