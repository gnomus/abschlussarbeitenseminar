\documentclass{beamer}

\usepackage[utf8]{inputenc}
\usepackage[T1]{fontenc}
\usepackage{microtype}
\usepackage[german]{babel}
\usepackage{hyperref}
\usepackage{lmodern}
\usepackage{wasysym}
\usepackage{textpos}
\usepackage{ulem}
\usepackage{epstopdf}
\usepackage[pageofpages=von,
            titlepagelogo=resource/logo]{beamerthemeUhh}
\usepackage{beamercolorthemeuhh}

\usepackage{tikz}
\usepackage{setspace}

\newenvironment{changemargin}[2]{%
\begin{list}{}{%
\setlength{\topsep}{0pt}%
\setlength{\leftmargin}{#1}%
\setlength{\rightmargin}{#2}%
\setlength{\listparindent}{\parindent}%
\setlength{\itemindent}{\parindent}%
\setlength{\parsep}{\parskip}%
}%
\item[]}{\end{list}}

\logo{\includegraphics[height=0.125\paperheight]{resource/logo-fbi}}

\setbeamertemplate{frametitle continuation}[from second][]
\usefonttheme{professionalfonts}

%%%%%
\newcommand{\myName}{Felix Favre}
\newcommand{\myEmail}{1favre@informatik.uni-hamburg.de}
\newcommand{\myTitle}{Validierung von dynamisch geladenem Javascript Code in Webapplikationen}
\newcommand{\myKeywords}{}
\institute{Universität Hamburg -- Fachbereich Informatik -- Abschlussarbeiten Seminar}
\date{\today}


\title{\myTitle}
\author{\myName}
\keywords{\myKeywords}
\subject{}

\definecolor{uhh}{RGB}{226,0,26}

\hypersetup{
	pdftitle={\myTitle},
	pdfauthor={\myName},
	pdfkeywords={\myKeywords},
	bookmarksnumbered=true,
	bookmarksopen=true,
	bookmarksopenlevel=1,
	colorlinks=true,
	linkbordercolor={1 1 1},
	urlcolor=uhh,
	anchorcolor=black,
	linkcolor=black,
	citecolor=black,
	filecolor=black,
	menucolor=black
}

\pdfcompresslevel=9
\pdfimageresolution=72
\pdfpkresolution=72

\newcommand{\maxFrameImage}[1]{
\begingroup
\setbeamercolor{background canvas}{bg=black}
\begin{frame}[plain]
\begin{changemargin}{-1cm}{-1cm}
\begin{center}
\includegraphics[width=\paperwidth,height=\paperheight,keepaspectratio]
{#1}
\end{center}
\end{changemargin}
\end{frame}
\endgroup
}

\begin{document}

\pdfbookmark[2]{Titelseite}{title}
{
    \usebackgroundtemplate{\includegraphics[width=\paperwidth]{resource/uhh-bg.png}}
    \frame{
        \titlepage
    }
}

\begin{frame}{Was geplant war}
  \begin{itemize}
    \item Motivation und Einleitung ausformulieren
    \item Gliederung weiter verfeinern
    \item Mehr Literatur raussuchen
    \item Weiter schreiben
  \end{itemize}
\end{frame}

\begin{frame}{Wozu ich gekommen bin}
\end{frame}

\begin{frame}{Zum Inhalt der Arbeit}
  \begin{itemize}
    \item Rahmenbedingung: Keine Verschlüsselte Verbingung
    \item Wie stelle ich sicher, dass eine Datei so beim Empfänger ankommt, wie ich sie verschickt habe?
    \item Mehrere Ideen
    \item Ideen in der Arbeit betrachten und evaluieren
  \end{itemize}
\end{frame}

\begin{frame}{Zum Inhalt der Arbeit - Idee 1}
  \begin{itemize}
    \item Server schickt eine Hashsumme des Files mit
    \item Anfällig für Man in the Middle
  \end{itemize}
  \begin{itemize}
    \item Server liefert File unter der Hashsumme aus
    \item \texttt{<script src="{}http://example.com/js/ d8e8fca2dc0f896fd7cb4cb0031ba249"{}></script>}
    \item Anfällig für Man in the Middle
  \end{itemize}
\end{frame}

\begin{frame}{Zum Inhalt der Arbeit - Idee 2}
  \begin{itemize}
    \item Server signiert Files mit privatem Schlüssel
    \item Öffentlicher Schlüssel wird mit dem File ausgeliefert
    \item Man in the Middle muss bloß beides fälschen
  \end{itemize}
\end{frame}

\begin{frame}{Zum Inhalt der Arbeit - Idee 3}
  \begin{itemize}
    \item Server signiert Files mit privatem Schlüssel
    \item Öffentlichen Schlüssel auf vertrauenswürdigem Server ablegen
    \item Selbe Problematik wie bei SSL
  \end{itemize}
\end{frame}

\begin{frame}{Zum Inhalt der Arbeit - Idee 4}
  \begin{itemize}
    \item Server signiert Files mit privatem Schlüssel
    \item Web-of-Trust ähnliches Konstrukt, wie bei PGP
    \item Eigentlich ganz cool, nutzt dann nur wieder keiner :(
  \end{itemize}
\end{frame}

\end{document}
